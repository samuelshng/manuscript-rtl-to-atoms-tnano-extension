\documentclass[11pt]{letter}

\usepackage[utf8]{inputenc}
\usepackage{geometry}
\geometry{margin=0.5in,includehead=false}
\usepackage{xcolor}
\usepackage[hidelinks]{hyperref}
\usepackage{booktabs}
\usepackage{tabularx}
\usepackage{enumitem}
\usepackage{caption}
\DeclareCaptionType{coverlettable}[Table][List of Tables]
\newenvironment{tableitemize}{%
  \begingroup
  \setlength{\parskip}{0pt}%
  \vspace{-0.55\baselineskip}%
  \begin{itemize}[leftmargin=*,nosep,topsep=0pt,partopsep=0pt]%
  \setlength{\itemsep}{0pt}%
}{%
  \end{itemize}%
  \endgroup
}
\usepackage{array}
\usepackage{multirow}
\usepackage{makecell}

\usepackage[acronym]{glossaries}

% Disable hyperlinks for glossary entries
\glsdisablehyper

% ML
\newacronym{ai}{AI}{artificial intelligence}
\newacronym{ml}{ML}{machine learning}
\newacronym{llm}{LLM}{large language model}
\newacronym{qat}{QAT}{quantization-aware training}
\newacronym{api}{API}{application programming interface}

% ML Accel
\newacronym{mvm}{MVM}{matrix--vector multiplication}
\newacronym{mxu}{MXU}{matrix multiply unit}
\newacronym{mac}{MAC}{multiply--accumulate}

% Atomic/Nanotech/Physics
\newacronym{stm}{STM}{scanning tunneling microscope}
\newacronym{afm}{AFM}{atomic force microscope}
\newacronym{set}{SET}{single-electron transistor}
\newacronym{bdl}{BDL}{binary-dot logic}

% Field-Coupled Nanocomputing
\newacronym{fcn}{FCN}{field-coupled nanocomputing}
\newacronym{sidb}{SiDB}{silicon dangling bond}
\newacronym{qca}{QCA}{quantum-dot cellular automata}
\newacronym{nml}{NML}{nanomagnetic logic}

% Devices
\newacronym{hal}{HAL}{hardware abstraction layer}
\newacronym{pi}{PI}{primary-input}
\newacronym{io}{I/O}{input/output}
\newacronym{adc}{ADC}{analog-to-digital converter}
\newacronym{cmos}{CMOS}{complementary metal-oxide-semiconductor}
\newacronym{cad}{CAD}{computer-aided design}
\newacronym{tpu}{TPU}{tensor processing unit}
\newacronym{alu}{ALU}{arithmetic logic unit}

% Architecture
\newacronym{pe}{PE}{processing element}

% EDA
\newacronym{rtl}{RTL}{register-transfer level}
\newacronym{eda}{EDA}{electronic design automation}
\newacronym{aig}{AIG}{And-Inverter Graph}

% fiction-specific algorithms
\newacronym{gold}{\emph{gold}}{graph-oriented layout design}
\newacronym{plo}{\emph{PLO}}{post-layout optimization}

% Others
\newacronym[longplural={figures of merit}]{fom}{FoM}{figure of merit}

\newcommand{\applyalwaysshortacros}{%
  \glsunset{ai}%
  \glsunset{ml}%
  \glsunset{llm}%
  \glsunset{cmos}%
  \glsunset{cad}%
}

\applyalwaysshortacros
% To-do items (kept identical to conference_101719.tex)
\newcommand{\TODO}[1]{\textcolor{red}{TODO: #1}}

\usepackage{newfloat}
\usepackage{float}

\DeclareFloatingEnvironment[fileext=lot]{table}

\indentedwidth=50em

\signature{Samuel~S.~H.~Ng\\
Ph.D.~Candidate\\
Department of Electrical and Computer Engineering\\
University of British Columbia\\
Vancouver, BC, Canada\\
\href{mailto:samueln@ece.ubc.ca}{samueln@ece.ubc.ca}}


\begin{document}

\date{November 30, 2025}

\begin{letter}{Professor Sorin Cotofana\\
Editor-in-Chief\\
IEEE Transactions on Nanotechnology}

\opening{Dear Professor Cotofana,}

This letter accompanies the manuscript ``RTL-to-Atoms Synthesis of a Machine Learning Accelerator on Atomic-Scale Computers'', submitted for consideration as a Regular Paper in IEEE Transactions on Nanotechnology (TNANO), Special Section associated with the 25th IEEE International Conference on Nanotechnology (IEEE NANO 2025). The work lies primarily in computational nanotechnology, with strong connections to circuits and architectures, and aligns with the NANO~2025 scope on quantum, neuromorphic, and unconventional computing.

The manuscript presents an end-to-end synthesis framework that maps a quantized matrix-multiply processing element for machine learning acceleration from \gls{rtl} descriptions down to dot-accurate \gls{sidb} layouts suitable for fabrication. By combining a hierarchical, parameterized \gls{rtl} architecture with \gls{sidb}-aware logic synthesis and physical design, the study shows how design decisions at the architecture, arithmetic, and layout levels jointly influence area, robustness, and scalability in atomic-scale computing platforms, thereby contributing to TNANO's focus on nanoscale devices, circuits, and systems.

This manuscript is a substantially extended journal version of our IEEE NANO 2025 conference paper ``Building a Machine Learning Accelerator with Silicon Dangling Bonds: From Verilog to Quantum Dot Layout'', which is properly mentioned and cited in the manuscript. A detailed comparison between the original and new contributions is provided in Table 1, highlighting substantial additions at every step of the synthesis pipeline. In particular, the journal version introduces parameterizable bit-widths for systematic scaling studies, \gls{sidb}-specific \gls{alu} implementations, and placement-and-routing cost functions tailored to \gls{sidb} logic, and a significantly expanded experimental evaluation across multiple precision settings.

The authors confirm that this submission represents original work that has not been published previously and is not under consideration for publication elsewhere. All co-authors have approved the manuscript and its submission to IEEE Transactions on Nanotechnology. The author order has been updated relative to the conference version to reflect additional contributions to the extended study.

Thank you very much for considering this manuscript for publication in the IEEE Transactions on Nanotechnology Special Section associated with the 25th IEEE International Conference on Nanotechnology. The authors appreciate the time and effort of the editors and reviewers in evaluating this work.

{\setlength{\indentedwidth}{\textwidth}\closing{Sincerely,}}


\begin{table}
    \small
    \caption{Summary of conference and journal contributions.}
    \label{tab:cover-letter-contribution-matrix}
    {\renewcommand{\arraystretch}{1.15}%
    \begin{tabularx}{\linewidth}{>{\raggedright\arraybackslash}p{0.11\linewidth} >{\raggedright\arraybackslash}X >{\raggedright\arraybackslash}X}
        \toprule
        \textbf{Topic} & \textbf{Conference contributions} & \textbf{New contributions for journal extension} \\
        \midrule
        \gls{rtl} design of MXU &
        \begin{tableitemize}
            \item Hierarchical \gls{rtl} for the matrix-multiply unit with a processing-element core and clocked shell.
            \item Each layer in the hierarchy verified with dedicated test benches.
            \item Fixed weight and activation precision at W8A8.
        \end{tableitemize} &
        \begin{tableitemize}
            \item Parameterizable weight and activation bit-widths that enable scaling studies and benchmarking of state-of-the-art placement-and-routing algorithms when 8-bit layouts become intractable.
        \end{tableitemize} \\
        \midrule
        \gls{rtl}-to-netlist &
        \begin{tableitemize}
            \item Used \emph{Yosys} for \gls{rtl}-to-AIG conversion with its default \gls{alu} mapping geared toward CMOS libraries.
            \item Optimized the AIG using ABC's \emph{\&deepsyn} strategy.
        \end{tableitemize} &
        \begin{tableitemize}
            \item Implemented \gls{sidb}-aligned ripple-carry adders and array multipliers as gate-level netlists and integrated them into \emph{Yosys} for technology-aware mapping.
        \end{tableitemize} \\
        \midrule
        Netlist-to-atoms &
        \begin{tableitemize}
            \item Applied figure-of-merit-aware technology mapping that favored robust \gls{sidb} gates despite the area overhead.
            \item Extended the hexagonalization flow to align input and output pins with the fabric clocking scheme.
        \end{tableitemize} &
        \begin{tableitemize}
            \item Added \gls{sidb}-specific cost objectives to the \emph{gold} placement-and-routing algorithm so that optimization targets the layout rules of \gls{sidb} fabrics rather than metrics tuned for other FCN platforms.
        \end{tableitemize} \\
        \midrule
        Experiment &
        \begin{tableitemize}
            \item Compared synthesized MXU layouts against manually estimated blueprints from prior studies.
            \item Evaluated uniform versus figure-of-merit-aware synthesis for the W8A8 configuration using the \emph{ortho} placement-and-routing algorithm.
        \end{tableitemize} &
        \begin{tableitemize}
            \item Compared synthesis results across W8A8, W4A4, and W2A2.
            \item Evaluated all bit-width configurations with figure-of-merit-aware technology mapping.
            \item Reported \emph{gold} placement-and-routing results for W4A4 and W2A2, which remain within \emph{gold}'s tractable range.
            \item Benchmarked new \emph{gold} cost objectives to quantify \gls{sidb}-specific layout improvements.
        \end{tableitemize} \\
        \bottomrule
    \end{tabularx}}
\end{table}

\end{letter}

\end{document}
